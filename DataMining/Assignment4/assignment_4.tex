%%%%%%%%%%%%%%%%%%%%%%%%%%%%%%%%%%%%%%%%%
% Structured General Purpose Assignment
% LaTeX Template
%
% This template has been downloaded from:
% http://www.latextemplates.com
%
% Original author:
% Ted Pavlic (http://www.tedpavlic.com)
%
% Note:
% The \lipsum[#] commands throughout this template generate dummy text
% to fill the template out. These commands should all be removed when 
% writing assignment content.
%
%%%%%%%%%%%%%%%%%%%%%%%%%%%%%%%%%%%%%%%%%

%----------------------------------------------------------------------------------------
%	PACKAGES AND OTHER DOCUMENT CONFIGURATIONS
%----------------------------------------------------------------------------------------

\documentclass{article}

\usepackage{fancyhdr} % Required for custom headers
\usepackage{lastpage} % Required to determine the last page for the footer
\usepackage{extramarks} % Required for headers and footers
\usepackage{graphicx} % Required to insert images
%\usepackage{lipsum} % Used for inserting dummy 'Lorem ipsum' text into the template
\usepackage{fourier} % Use the Adobe Utopia font for the document - comment this line to return to the LaTeX default  Margins
%\usepackage{cite}
\usepackage{hyperref}
%\usepackage[framed]{mcode}
%\usepackage{listings}
\usepackage{color}
\usepackage{xcolor}
\usepackage{amsmath}
\usepackage{float}
\usepackage{framed}
%\usepackage{tcolorbox}
\definecolor{shadecolor}{RGB}{180,180,180}

\topmargin=-0.45in
\evensidemargin=0in
\oddsidemargin=0in
\textwidth=6.5in
\textheight=9.0in
\headsep=0.25in 

\linespread{1.1} % Line spacing

% Set up the header and footer
\pagestyle{fancy}
%\lhead{\hmwkAuthorName} % Top left header
%\chead{\hmwkClass\ (\hmwkClassInstructor\ \hmwkClassTime): \hmwkTitle} % Top center header
%\rhead{\firstxmark} % Top right header
%\lfoot{\lastxmark} % Bottom left footer
\cfoot{} % Bottom center footer
\rfoot{ \thepage\ } % Bottom right footer
\renewcommand\headrulewidth{0.4pt} % Size of the header rule
\renewcommand\footrulewidth{0.4pt} % Size of the footer rule

\setlength\parindent{0pt} % Removes all indentation from paragraphs

%----------------------------------------------------------------------------------------
%	DOCUMENT STRUCTURE COMMANDS
%	Skip this unless you know what you're doing
%----------------------------------------------------------------------------------------

% Header and footer for when a page split occurs within a problem environment
\newcommand{\enterProblemHeader}[1]{
\nobreak\extramarks{#1}{#1 continued on next page\ldots}\nobreak
\nobreak\extramarks{#1 (continued)}{#1 continued on next page\ldots}\nobreak
}

% Header and footer for when a page split occurs between problem environments
\newcommand{\exitProblemHeader}[1]{
\nobreak\extramarks{#1 (continued)}{#1 continued on next page\ldots}\nobreak
\nobreak\extramarks{#1}{}\nobreak
}

\setcounter{secnumdepth}{0} % Removes default section numbers
\newcounter{homeworkProblemCounter} % Creates a counter to keep track of the number of problems

\newcommand{\homeworkProblemName}{}
\newenvironment{homeworkProblem}[1][Problem \arabic{homeworkProblemCounter}]{ % Makes a new environment called homeworkProblem which takes 1 argument (custom name) but the default is "Problem #"
\stepcounter{homeworkProblemCounter} % Increase counter for number of problems
\renewcommand{\homeworkProblemName}{#1} % Assign \homeworkProblemName the name of the problem
\section{\homeworkProblemName} % Make a section in the document with the custom problem count
\enterProblemHeader{} % Header and footer within the environment
}{
\exitProblemHeader{} % Header and footer after the environment
}

%\newcommand{\problemAnswer}[1]{ % Defines the problem answer command with the content as the only argument
%\noindent\framebox[\columnwidth][c]{\begin{minipage}{0.98\columnwidth}#1\end{minipage}} % Makes the box around the problem answer and puts the content inside
%}

\newcommand{\homeworkSectionName}{}
\newenvironment{homeworkSection}[1]{ % New environment for sections within homework problems, takes 1 argument - the name of the section
\renewcommand{\homeworkSectionName}{#1} % Assign \homeworkSectionName to the name of the section from the environment argument
\subsection{\homeworkSectionName} % Make a subsection with the custom name of the subsection
\enterProblemHeader{} % Header and footer within the environment
}{
\enterProblemHeader{} % Header and footer after the environment
}
   
%----------------------------------------------------------------------------------------
%	NAME AND CLASS SECTION
%----------------------------------------------------------------------------------------

\newcommand{\hmwkTitle}{Assignment\ \#4} % Assignment title
%\newcommand{\hmwkDueDate}{Monday,\ January\ 1,\ 2012} % Due date
%\newcommand{\hmwkClass}{BIO\ 101} % Course/class
%\newcommand{\hmwkClassTime}{10:30am} % Class/lecture time
%\newcommand{\hmwkClassInstructor}{Jones} % Teacher/lecturer
\newcommand{\hmwkAuthorName}{Akhil P M (SC14M044)} % Your name

\newcommand{\tab}[1]{\hspace{.1\textwidth}\rlap{#1}}


%----------------------------------------------------------------------------------------
%	TITLE PAGE
%----------------------------------------------------------------------------------------

\title{
\textsc{Indian Institute of Space Science and Technology Thiruvananthapuram} \\ [25pt]
\vspace{2in}
\textmd{\textbf{\hmwkTitle}}\\
\normalsize\vspace{0.1in}\small{Due\ on\ 05-11-2014}\\
%\vspace{0.1in}\large{\textit{\hmwkClassInstructor\ \hmwkClassTime}}
\vspace{3in}
}

\author{\textbf{\hmwkAuthorName}}
\date{} % Insert date here if you want it to appear below your name

%----------------------------------------------------------------------------------------

\begin{document}

\maketitle

%----------------------------------------------------------------------------------------
%	TABLE OF CONTENTS
%----------------------------------------------------------------------------------------

%\setcounter{tocdepth}{1} % Uncomment this line if you don't want subsections listed in the ToC

\newpage
\tableofcontents
\newpage

%----------------------------------------------------------------------------------------
%	PROBLEM 1
%----------------------------------------------------------------------------------------

% To have just one problem per page, simply put a \clearpage after each problem

\begin{homeworkProblem}[\textbf{1.Ridge Regression}]

\begin{homeworkSection}{\textbf{1.1 2nd degree polynomial fitting }}
$\lambda$ v/s training \& validation error.

\begin{figure}[H]
	\centering
	\includegraphics{xpow2fit}
	\caption{fitting 2nd degree polynomial}
\end{figure}

Plot of J\_reg(W) 
\begin{figure}[H]
	\centering
	\includegraphics{jregwPow2}
	\caption{J(W) v/s iterations}
\end{figure}

\end{homeworkSection}

\begin{homeworkSection}{\textbf{1.2 3rd degree polynomial fitting }}
$\lambda$ v/s training \& validation error.

\begin{figure}[H]
	\centering
	\includegraphics{xpow3fit}
	\caption{fitting 3rd degree polynomial}
\end{figure}
Plot of J\_reg(W) 
\begin{figure}[H]
	\centering
	\includegraphics{jregwPow3}
	\caption{J(W) v/s iterations}
\end{figure}

\end{homeworkSection}

\begin{homeworkSection}{\textbf{1.3 7th degree polynomial fitting }}
$\lambda$ v/s training \& validation error.

\begin{figure}[H]
	\centering
	\includegraphics{xpow7fit}
	\caption{fitting 7th degree polynomial}
\end{figure}
Plot of J\_reg(W) 
\begin{figure}[H]
	\centering
	\includegraphics{jregwPow7}
	\caption{J(W) v/s iterations}
\end{figure}

\end{homeworkSection}

\begin{homeworkSection}{\textbf{1.4 Performance Comparisons }}

best model with 2nd degree polynomial $\colon$ $\lambda$=0.10 test error estimated=1.2096\\
best model with 3rd degree polynomial $\colon$ $\lambda$=0.40 test error estimated=1.1777\\
best model with 7th degree polynomial $\colon$ $\lambda$=2.00 test error estimated=1.1389\\
performance of the least square method $\colon$ 1.4270 

Thus 7th degree polynomial gives the best fit for the data.
\end{homeworkSection}

\end{homeworkProblem}

\begin{homeworkProblem}[\textbf{2.Regularised linear regression}]

Weight values(without regularisation) 
   $\theta_0$ = 11.6506  $\theta_1$ = -1.7925 $\theta_2$ = 4.4062 
   $\theta_3$ = -1.6779 $\theta_4$ = 4.1600 $\theta_5$ = -0.5733 $\theta_6$ = 16.4432 
   $\theta_7$ = 1.6647 $\theta_8$ = 0.3262 $\theta_9$ = 0.9794 $\theta_{10}$ = -2.1768
   $\theta_{11}$ = -4.6693 $\theta_{12}$ = 8.5775 $\theta_{13}$ = -10.4561 \\

Weight values(with regularisation) 
   $\theta_0$ = 11.9142  $\theta_1$ = -1.8099 $\theta_2$ = 4.4106 
   $\theta_3$ = -1.6696 $\theta_4$ = 4.1478 $\theta_5$ = -0.5509 $\theta_6$ = 15.9588 
   $\theta_7$ = 1.7438 $\theta_8$ = 0.3979 $\theta_9$ = 0.9771 $\theta_{10}$ = -2.1800
   $\theta_{11}$ = -4.6242 $\theta_{12}$ = 8.5004 $\theta_{13}$ = -10.6712 \\

cost witout regularisation$\colon$8.37 \\
cost with regularisation$\colon$8.20 \\
$\alpha$ value $\colon$0.500000 \\
$\lambda$ value $\colon$0.10 \\

\end{homeworkProblem}

\begin{homeworkProblem}[\textbf{3.K-nearest neighbourhood}]

no of test datasets$\colon$209 \\
no of folds$\colon$5 \\
no of misclassifications$\colon$69 \\
accuracy$\colon$0.670 \\
precision$\colon$0.462 \\
recall/sensitivity$\colon$0.369 \\
F-Measure$\colon$0.410 \\
Max accuracy during cross validation$\colon$0.617 \\
optimum K value$\colon$25\\

The ROC-curve obtained is shown below
\begin{figure}[H]
	\centering
	\includegraphics{ROCpgm3withK25}
	\caption{ROC curve for k=25}
\end{figure}

\end{homeworkProblem}

\begin{homeworkProblem}[\textbf{4.Decision Tree in WEKA}]

=== Evaluation on test split === \\

Correctly Classified Instances$\colon$5584 (85.7494\%) \\
Incorrectly Classified Instances$\colon$928 (14.2506\%) \\
Kappa statistic$\colon$0.5732 \\
K\&B Relative Info Score$\colon$288671.6144 \\
K\&B Information Score$\colon$2308.5563 bits \tab{0.3545 bits/instance}\\
Class complexity | order 0 $\colon$ 5100.5935 bits \tab{0.7833 bits/instance}\\
Class complexity | scheme $\colon$ 40494.5293 bits \tab{6.2184 bits/instance}\\
Complexity improvement(Sf)$\colon$-35393.9358 bits \tab{-5.4352 bits/instance}\\
Mean absolute error$\colon$0.1917 \\
Root mean squared error$\colon$0.3191 \\
Relative absolute error$\colon$52.867\% \\
Root relative squared error$\colon$75.4903\% \\
Total Number of Instances$\colon$6512 \\    

=== Detailed Accuracy By Class ===\\ \\
\begin{tabular}{|l|l|l|l|l|l|l|}
	\hline
	\textbf{TP Rate} & \textbf{FP Rate} & \textbf{Precision} &\textbf{Recall} &\textbf{F\-Measure} &\textbf{ROC Area} &\textbf{Class}\\
    0.936 & 0.4 & 0.885 & 0.936 & 0.91 & 0.89 & <=50K\\
    0.6 & 0.064 & 0.739 & 0.6 & 0.662 & 0.89 & >50K\\
    \hline
    \multicolumn{7}{|c|}{Weighted Avg}\\
    \hline
    0.857 & 0.322 & 0.851 & 0.857 & 0.852 & 0.89 & -\\ 
    \hline
\end{tabular}\\ \\
=== Confusion Matrix ===\\ \\
\begin{tabular}{|l|l|}
	\hline
	\textbf{Class<=50k} & \textbf{Class>50k}\\
	4674 & 322\\
  	606 & 910\\
  	\hline
\end{tabular}\\ \\

\textbf{The Complete set of results with the obtained decision tree is accessible in this \href{https://github.com/akhilpm/M.tech-Assignments/blob/master/DataMining/Assignment4/wekaDecisionTree.txt}{link}}(since it is around 1000 lines it is not included here)

\end{homeworkProblem}

\begin{homeworkProblem}[\textbf{5.Problem on Apriori Algorithm}]
Support counts of individuals\\ \\
\begin{tabular}{|l|l|l|l|l|l|l|l|l|l|l|}
	\hline
	M & O & N & K & E & Y & D & A & U & C & I\\
	\hline
	3 & 4 & 2 & 5 & 4 & 3 & 1 & 1 & 1 & 2 & 1\\
	\hline
\end{tabular}\\ \\

Since Min.Support is 3(60\%) we form L1 as\\ \\
\begin{tabular}{|l|l|l|l|l|}
	\hline
	M & O & K & E & Y\\
	\hline
	3 & 4 & 5 & 4 & 3\\ 
	\hline
\end{tabular}\\ \\

Then C2 is formed as shown below\\ \\
\begin{tabular}{|l|l|l|l|l|l|l|l|l|l|l|}
	\hline
	{M,O} & {M,K} & {M,E} & {M,Y} & {O,K} & {O,E} & {O,Y} & {K,E} & {K,Y} & {E,Y}\\
	\hline
	1 & 3 & 2 & 2 & 4 & 4 & 2 & 4 & 3 & 2\\
	\hline
\end{tabular}\\ \\

L2 obtained is\\ \\
\begin{tabular}{|l|l|l|l|l|}
	\hline
	{M,K} & {O,K} & {O,E} & {K,E} & {K,Y}\\
	\hline
	3 & 4 & 4 & 4 & 3\\ 
	\hline
\end{tabular}\\ \\

Then C3 is formed as shown below\\
\begin{tabular}{|l|}
	\hline
	{M,K,E}\\
	\hline
	4\\
	\hline
\end{tabular}\\ \\

C3 is same as L3 since its support count is 4(>=3).\\
The rules formed are
\begin{snugshade*}
	\{O,K\} $\Longrightarrow$ E\\
	\{O,E\} $\Longrightarrow$ K\\
	\{K,E\} $\Longrightarrow$ O\\
\end{snugshade*}

$\frac{O,K,E}{O,K}$ = $\frac{4}{4}$ = 1\\
$\frac{O,K,E}{O,E}$ = $\frac{4}{4}$ = 1\\
$\frac{O,K,E}{K,E}$ = $\frac{4}{4}$ = 1\\
\\
since confidence >=80\%, all are strong associations
\end{homeworkProblem}

\end{document}
