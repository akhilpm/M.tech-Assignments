%%%%%%%%%%%%%%%%%%%%%%%%%%%%%%%%%%%%%%%%%
% Structured General Purpose Assignment
% LaTeX Template
%
% This template has been downloaded from:
% http://www.latextemplates.com
%
% Original author:
% Ted Pavlic (http://www.tedpavlic.com)
%
% Note:
% The \lipsum[#] commands throughout this template generate dummy text
% to fill the template out. These commands should all be removed when 
% writing assignment content.
%
%%%%%%%%%%%%%%%%%%%%%%%%%%%%%%%%%%%%%%%%%

%----------------------------------------------------------------------------------------
%	PACKAGES AND OTHER DOCUMENT CONFIGURATIONS
%----------------------------------------------------------------------------------------

\documentclass{article}

\usepackage{fancyhdr} % Required for custom headers
\usepackage{lastpage} % Required to determine the last page for the footer
\usepackage{extramarks} % Required for headers and footers
\usepackage{graphicx} % Required to insert images
\usepackage{lipsum} % Used for inserting dummy 'Lorem ipsum' text into the template
\usepackage{fourier} % Use the Adobe Utopia font for the document - comment this line to return to the LaTeX default  Margins
%\usepackage{cite}
\usepackage{hyperref}
%\usepackage[framed]{mcode}
%\usepackage{listings}
\usepackage{color}
\usepackage{xcolor}
\usepackage{amsmath}
\usepackage{float}
%\usepackage{framed}
\usepackage{amsfonts}
%\usepackage{tcolorbox}
\definecolor{shadecolor}{RGB}{180,180,180}

\topmargin=-0.45in
\evensidemargin=0in
\oddsidemargin=0in
\textwidth=6.5in
\textheight=9.0in
\headsep=0.25in 

\linespread{1.1} % Line spacing

% Set up the header and footer
\pagestyle{fancy}
%\lhead{\hmwkAuthorName} % Top left header
%\chead{\hmwkClass\ (\hmwkClassInstructor\ \hmwkClassTime): \hmwkTitle} % Top center header
%\rhead{\firstxmark} % Top right header
%\lfoot{\lastxmark} % Bottom left footer
\cfoot{} % Bottom center footer
\rfoot{ \thepage\ } % Bottom right footer
\renewcommand\headrulewidth{0.4pt} % Size of the header rule
\renewcommand\footrulewidth{0.4pt} % Size of the footer rule

\setlength\parindent{0pt} % Removes all indentation from paragraphs

%----------------------------------------------------------------------------------------
%	DOCUMENT STRUCTURE COMMANDS
%	Skip this unless you know what you're doing
%----------------------------------------------------------------------------------------

% Header and footer for when a page split occurs within a problem environment
\newcommand{\enterProblemHeader}[1]{
\nobreak\extramarks{#1}{#1 continued on next page\ldots}\nobreak
\nobreak\extramarks{#1 (continued)}{#1 continued on next page\ldots}\nobreak
}

% Header and footer for when a page split occurs between problem environments
\newcommand{\exitProblemHeader}[1]{
\nobreak\extramarks{#1 (continued)}{#1 continued on next page\ldots}\nobreak
\nobreak\extramarks{#1}{}\nobreak
}

\setcounter{secnumdepth}{0} % Removes default section numbers
\newcounter{homeworkProblemCounter} % Creates a counter to keep track of the number of problems

\newcommand{\homeworkProblemName}{}
\newenvironment{homeworkProblem}[1][Problem \arabic{homeworkProblemCounter}]{ % Makes a new environment called homeworkProblem which takes 1 argument (custom name) but the default is "Problem #"
\stepcounter{homeworkProblemCounter} % Increase counter for number of problems
\renewcommand{\homeworkProblemName}{#1} % Assign \homeworkProblemName the name of the problem
\section{\homeworkProblemName} % Make a section in the document with the custom problem count
\enterProblemHeader{} % Header and footer within the environment
}{
\exitProblemHeader{} % Header and footer after the environment
}

%\newcommand{\problemAnswer}[1]{ % Defines the problem answer command with the content as the only argument
%\noindent\framebox[\columnwidth][c]{\begin{minipage}{0.98\columnwidth}#1\end{minipage}} % Makes the box around the problem answer and puts the content inside
%}

\newcommand{\homeworkSectionName}{}
\newenvironment{homeworkSection}[1]{ % New environment for sections within homework problems, takes 1 argument - the name of the section
\renewcommand{\homeworkSectionName}{#1} % Assign \homeworkSectionName to the name of the section from the environment argument
\subsection{\homeworkSectionName} % Make a subsection with the custom name of the subsection
\enterProblemHeader{} % Header and footer within the environment
}{
\enterProblemHeader{} % Header and footer after the environment
}
   
%----------------------------------------------------------------------------------------
%	NAME AND CLASS SECTION
%----------------------------------------------------------------------------------------

\newcommand{\hmwkTitle}{Assignment\ \#1} % Assignment title
%\newcommand{\hmwkDueDate}{Monday,\ January\ 1,\ 2012} % Due date
%\newcommand{\hmwkClass}{BIO\ 101} % Course/class
%\newcommand{\hmwkClassTime}{10:30am} % Class/lecture time
%\newcommand{\hmwkClassInstructor}{Jones} % Teacher/lecturer
\newcommand{\hmwkAuthorName}{Akhil P M (SC14M044)} % Your name

\newcommand{\tab}[1]{\hspace{.05\textwidth}\rlap{#1}}


%----------------------------------------------------------------------------------------
%	TITLE PAGE
%----------------------------------------------------------------------------------------

\title{
\textsc{Indian Institute of Space Science and Technology Thiruvananthapuram} \\ [25pt]
\vspace{2in}
\textmd{\textbf{\hmwkTitle}}\\
\normalsize\vspace{0.1in}\small{Due\ on\ 02-02-2015}\\
%\vspace{0.1in}\large{\textit{\hmwkClassInstructor\ \hmwkClassTime}}
\vspace{3in}
}

\author{\textbf{\hmwkAuthorName}}
\date{} % Insert date here if you want it to appear below your name

%----------------------------------------------------------------------------------------

\begin{document}

\maketitle

%----------------------------------------------------------------------------------------
%	TABLE OF CONTENTS
%----------------------------------------------------------------------------------------

%\setcounter{tocdepth}{1} % Uncomment this line if you don't want subsections listed in the ToC

\newpage
\tableofcontents
\newpage

%----------------------------------------------------------------------------------------
%	PROBLEM 1
%----------------------------------------------------------------------------------------

% To have just one problem per page, simply put a \clearpage after each problem

\begin{homeworkProblem}[\textbf{Metric Space,Normed Space,Vector Space}]

\begin{homeworkSection}{\textbf{1. Show that the set \textit{X} of all integers with metric defined by d(m,n) = |m-n| is a complete metric space.}}
A sequence $x_1$,$x_2$,$x_3$$\ldots$ is called a Caushy`s sequence if\\ 
$\forall \epsilon>0$ \tab{$\exists N |$  $\forall m,n>N$ $d(x_m,x_n)<\epsilon$}

Take $\epsilon$=$\frac{1}{2}$.\\
Then $\exists N$ $\colon$ $\forall m,n>N$ $d(x_m,x_n)<\frac{1}{2}$\\
Since X is the set of integers and the difference cannot exceed $\frac{1}{2}$, then it should be 0; ie if a metric d is defined then $d(x_m,x_n)$ should be zero if $\epsilon$=$\frac{1}{2}$. 
\[ \Longrightarrow d(x_m,x_n) = |m-n| = 0\]
\[ \Longrightarrow x_m=x_n   \forall m,n>N \] 

Thus the sequence $\{x_n\}$ $\longrightarrow$ x when n>N. Hence all cauchy`s sequences converges,so $(X,d)$ is a complete metric space.
\end{homeworkSection}

\begin{homeworkSection}{\textbf{2. Show that d(x,y) = $\sqrt{|x-y|}$ defines a metric on the set of all real numbers}}
\[ d(x,y) = \sqrt{|x-y|} \forall x,y \in  \mathbb{R} \] 
To check that whether d is a metric, we need to verify all the axioms of a metric.\\
a. Since |x-y| > 0 $\forall x,y \in  \mathbb{R}$ d(x,y)>0 always.\\
   and d(x,y)=0 only when |x-y|=0 $\Longrightarrow$ x=y.\\ \\
b. \[\forall x,y \in  \mathbb{R} |x-y|=|y-x|\]
   \[\Longrightarrow d(x,y) =  \sqrt{|x-y|} =  \sqrt{|y-x|} = d(y,x)\]
c.Triangular inequality\\
  it states that $\forall x,y,z \in \mathbb{R}$
  \[ \sqrt{|x-z|} \leq \sqrt{|x-y|}+\sqrt{|y-z|} \] 	

  we know that,
  \[ |x-z| = |x-y+y-z| \leq |x-y|+|y-z| \]\\
  Thus it follows from the properties of square roots and the above inequality
  \[ \sqrt{|x-z|} \leq \sqrt{|x-y|}+\sqrt{|y-z|} \]
Hence all the axioms are satisfied. So d(x,y) = $\sqrt{|x-y|}$ is a metric in $\mathbb{R}.$  
\end{homeworkSection}

\begin{homeworkSection}{\textbf{3. Show that the closure $\overline{Y}$ of a subspace Y of a normed space X is again a vector space.}}

Inorder to prove that $\overline{Y}$ is a vector space it is sufficient to establish that
\[ \alpha x+\beta y \in \overline{Y} \forall x,y \in  \overline{Y} \]
and $\alpha \& \beta$ are from the underlying field \textit{F}.\\
We know that $0 \in \overline{Y}$ since $Y \subset \overline{Y}$. Since x,y $\in \overline{Y}$ there exists $x_i,y_i \in X$ such that $x_i \longrightarrow x$ and $y_i \longrightarrow y$.Since multiplication and addition are continuous,
\[ \alpha x_i+\beta y_i \longrightarrow \alpha x+\beta y \]
Therefore,$\alpha x+\beta y \in \overline{Y} $   
\end{homeworkSection}


\begin{homeworkSection}{\textbf{4. Show that in an inner product space, x$\perp$y iff $||x+\alpha y|| \geq ||x|| \forall \alpha \in \mathbb{R}$}}
we know that, 
\[ ||x+\alpha y||^2 = \langle x+\alpha y,x+\alpha y \rangle \]
\[ = \langle x,x \rangle + \alpha \langle x,y \rangle + \alpha \langle y,x \rangle + \alpha^2 \langle y,y \rangle \]
Assuming the underlying field to be $\mathbb{R}$, the inner product becomes symmetric, and we obtain
\[ ||x+\alpha y||^2 = \langle x,x \rangle + 2*\alpha \langle x,y \rangle + \alpha^2 \langle y,y \rangle \]

If x$\perp$y then $\langle x,y \rangle $=0. Thus 
\[ ||x+\alpha y||^2 = ||x||^2 + \alpha^2 ||y||^2 \]
\[\Longrightarrow ||x+\alpha y||^2 \geq ||x||^2 \]
\[\Longrightarrow ||x+\alpha y|| \geq ||x|| \]
since, $\alpha^2 ||y||^2$ is always a positive value. This will violate only when the following two conditions occur simultaneously.\\
I) x is not perpendicular to y\\
II) $2*\alpha \langle x,y \rangle \geq -\alpha^2 ||y||^2$ \\
Thus only if part is also verified.

\end{homeworkSection}

\begin{homeworkSection}{\textbf{5. Find $\langle u,v\rangle$, where v=$(1+2i,3-i)^T$, $u=(-2+i,4)^T$}}

\[ \langle u,v\rangle = \langle(-2+i,4),(1+2i,3-i)\rangle\]
for complex numbers $\langle (x1,x2)(y1,y2) \rangle = x1*\overline{y1}+x2*\overline{y2}$

\[= (-2+i)(1-2i) + 4(3+i)\]
\[= -2+4i+i+2+12+4i\]
\[= 9i+12\]

\end{homeworkSection}

\begin{homeworkSection}{\textbf{6. Which of the following subsets of $\mathbb{R}^3$ constitute a subspace of $\mathbb{R}^3$ ? [x=$(\eta_1,\eta_2,\eta_3)^T$]}}
(a) \textbf{All x with $\eta_1 = \eta_2$ and $\eta_3 = 0$.}\\
(b) \textbf{All x with $\eta_1 = \eta_2+1$}

a)\\
Let Z=\{All x with $\eta_1 = \eta_2$ and $\eta_3 = 0$\}.\\
Consider X=$(x,x,0)$,Y=$(y,y,0)$ $\in Z$
\[X+Y = (x+y,x+y,0) \in Z \]
\[\alpha X = (\alpha x,\alpha x,0) \in Z \]
Thus Z is closed under addition and scalar multiplication,hence it is a subspace of $\mathbb{R}^3$.

b)\\
Let Z=\{All x with $\eta_1 = \eta_2+1$\}.\\
Consider X=$(x+1,x,p)$,Y=$(y+1,y,q)$ $\in Z$ where p,q$\in \mathbb{R}$.
\[X+Y = (x+y+2,x+y,p+q)) \notin Z  \]
because $\eta_1 \neq \eta_2+1$ is violated here. Hence Z is not closed under addition. So it is not a subspace of $\mathbb{R}^3$.

\end{homeworkSection}

\begin{homeworkSection}{\textbf{7. Show that the norm ||x|| is the distance from x to 0}}
Every normed space is a metric space or norm induces a metric on a vector space. Thus in a metric space with an induced norm
\[ d(x,y) = ||x-y|| \]
We know that d(x,y) is, 
\[ d\colon X*X \longrightarrow \mathbb{K} \]
and norm is,
\[ ||\cdot|| \colon X \longrightarrow \mathbb{K} \]
where $\mathbb{K}$ is the underlying field.\\
if ||x|| is a metric in a metric space then we have, 
\[ d(x,y) = ||x|| \]
\[ \Longrightarrow y=0\]
It implies that we are calculating the distance from origin. Hence ||x|| is the distance from 0.
\end{homeworkSection}

\begin{homeworkSection}{\textbf{8. If in an inner product space $\langle x,u\rangle = \langle x,v\rangle$ for all x , show that u=v.}}
Since $\langle x,u\rangle = \langle x,v\rangle$,
\[ \langle x,u\rangle - \langle x,v\rangle = \langle x,u-v\rangle = 0\]
But we know that inner product is zero only when one of the two vectors is zero.(orthogonality case can be avoided, since x is neither orthogonal to u,nor to v, hence it cannot be orthogonal to a linear combination of u \& v.)\\
Here x cannot be zero $\forall$ x.
\[ \Longrightarrow u-v=0 \]
\[ \Longrightarrow u=v \]

\end{homeworkSection}

\begin{homeworkSection}{\textbf{9. Prove that $||T_1T_2|| \leq ||T_1||*||T_2||; ||T^n|| \leq ||T||^n$}}

This property is called submultiplicative property and is only valid for matrix norms.\\
An induced matrix norm ||T|| is defined as 
\[ ||T|| = \max \limits _{x \ne 0} \frac{||Tx||}{||x||} \] 
Thus $||T_1T_2||$ is 
\[ ||T_1T_2|| = \max \limits _{x \ne 0} \frac{||T_1T_2x||}{||x||} \]
\[ = \max \limits _{x \ne 0} \frac{||T_1T_2x||}{||T_2x||}\frac{||T_2x||}{||x||} \]
Putting $T_2x = y$ in the first part
\[ \leq \max \limits _{y \ne 0} \frac{||T_1y||}{||y||} * \max \limits _{x \ne 0} \frac{||T_2x||}{||x||} \]
\[ \leq ||T_1||*||T_2||\]

b)\\
\[ ||T^n|| \leq ||T||*||T \ldots T|| \]
\[ \leq ||T||*||T||*||T \ldots T|| \]
\[ \leq ||T||*||T||*\ldots*||T|| \]
\[ \leq ||T||^n \]
\end{homeworkSection}

\begin{homeworkSection}{\textbf{10. For a real inner product space prove that $\langle x,y \rangle$ = $\frac{1}{4}(||x+y||^2-||x-y||^2 )$}}
We know that,
\[ ||x+y||^2 = \langle x+y,x+y \rangle \]
\[ = \langle x,x \rangle + \langle x,y \rangle + \langle y,x \rangle + \langle y,y \rangle \]
but for real IPS $\langle x,y \rangle = \langle y,x \rangle$.also $ \langle x,x \rangle = ||x||^2$,Then,
\[ ||x+y||^2 = ||x||^2 + 2*\langle x,y \rangle + ||y||^2 \]
Similarly,
\[ ||x-y||^2 = ||x||^2 - 2*\langle x,y \rangle + ||y||^2 \]
\[ ||x+y||^2 - ||x-y||^2 = 4*\langle x,y \rangle\]
Thus,
\[ \langle x,y \rangle = \frac{1}{4}(||x+y||^2 - ||x-y||^2) \]
\end{homeworkSection}

\begin{homeworkSection}{\textbf{11. Define T$\colon \mathbb{R}^2 \longrightarrow \mathbb{R}^2$ by T(x,y)=(x,0). Is T a linear operator ?}}
An operator is said to be linear if\\
a)D(T) and R(T) are vector spaces over the same field $\mathbb{K}$.\\
b)T(x+y) = T(x)+T(y)\\
  $T(\alpha x) = \alpha T(x)$\\


In this case,D(T) = $\mathbb{R}^2$ is a vector space.\\
R(T) = (x,0) where x$\in \mathbb{R}$ is also a vector space. Let X,Y $\in \mathbb{R}^2$\\
\[T(X+Y) = T((x_1,x_2) + (y_1,y_2)) \]
\[ = T(x_1+y_1,x_2+y_2) \]
\[ = (x_1+y_1,0) = (x_1,0) + (y_1,0) \]
\[ = T(X)+T(Y) \]
Checking for the other condition,
\[T(\alpha X) = T(\alpha x_1,\alpha x_2) \]
\[ = (\alpha x_1,0) = \alpha(x_1,0) \]
\[ = \alpha T(X) \]
All the conditions are satisfied. Hence T is a linear operator. 
\end{homeworkSection}

\begin{homeworkSection}{\textbf{12. Show that a discrete metric space is complete.}}
Discrete metric $\rho$ on a set X is defined by,\\
\[
	\rho(x,y)=
	\begin{cases}
		1 & \text{if} x \neq y\\
		0 & \text{if} x = y
	\end{cases}
\]
for any x,y $\in X$. Here $(X,\rho)$ is a disrete metric space.\\
A sequence $x_1$,$x_2$,$x_3$$\ldots$ is called a Caushy`s sequence if\\ 
$\forall \epsilon>0$ \tab{$\exists N |$  $\forall m,n>N$ $d(x_m,x_n)<\epsilon$}

Take $\epsilon$=$\frac{1}{2}$.\\
Then $\exists N$ $\colon$ $\forall m,n>N$ $d(x_m,x_n)<\frac{1}{2}$\\
But possible values of d are \{0,1\}. Since distance cannot exceed $\frac{1}{2}$, $d(x_m,x_n)$ should be zero.
\[ \Longrightarrow d(x_m,x_n) = 0 \forall m,n>N \]
\[ \Longrightarrow x_n = x_m \]
Thus \{$x_n$\} $\longrightarrow x $ $\forall n>N$. This means every Caushy`s sequence converges in X. So discrete metric space is complete. 
\end{homeworkSection}

\begin{homeworkSection}{\textbf{13. Descibe Weistrass appoximation theorem.}}
\textbf{Theorem$\colon$} if \textit{f} is a continuous real valued function on \textit{[a,b]} and if any $\epsilon > 0$ is given, then there exists a polynomial \textit{p} on \textit{[a,b]} such that 
\[ |f(x)-p(x)| < \epsilon \]
for all x in [a,b]. In words, any continuous function on a closed and bounded interval can be uniformly approximated on that interval by polynomials to any degree of accuracy\cite{weistrass}.\\

Because polynomials are among the simplest functions, and because computers can directly evaluate polynomials, this theorem has both practical and theoretical relevance, especially in polynomial interpolation\cite{wiki}.As a consequence of the Weierstrass approximation theorem, one can show that the space $C[a,b]$ is separable$\colon$ the polynomial functions are dense, and each polynomial function can be uniformly approximated by one with rational coefficients; there are only countably many polynomials with rational coefficients. 

\end{homeworkSection}

\end{homeworkProblem}
%----------------------------------------------------------------------------------------

%----------------------------------------------------------------------------------------
\begin{thebibliography}{9}

\bibitem{weistrass}
  Weisstein, Eric W. ''Weierstrass Approximation Theorem.'' From MathWorld--A Wolfram Web Resource. \emph{http://mathworld.wolfram.com/WeierstrassApproximationTheorem.html}

\bibitem{wiki}
  Stone-Weierstrass theorem, \emph{http://en.wikipedia.org/wiki/Stone-Weierstrass\_theorem}

\end{thebibliography}


\end{document}
